\documentclass{article}

\usepackage{graphicx}
\usepackage{hyperref}
\usepackage{natbib}
\usepackage{apalike}

\title{First decoding of a rat locations from hippocampal spiking activity}

\author{Joaquin Rapela\thanks{j.rapela@ucl.ac.uk}}

\begin{document}

\maketitle

\section{Behavior}

A rat was run in the three-arm mazed depicted in Figure~\ref{fig:pos2D}.

\begin{figure}
    \begin{center}
        \href{https://www.gatsby.ucl.ac.uk/~rapela/hippocampalDecoding/figures/pos2D.html}{\includegraphics[width=5in]{../../../replay_trajectory_classification_test/figures/pos2D.png}}
		\caption{Positions occupied by the subject rat. Click on the figure to access its interactive version.}
        \label{fig:pos2D}
    \end{center}
\end{figure}

\section{Neural recordings}

We used recordings from 104 hippocampal units of rat. Data provided by Dr.~Denovellis. Figure~\ref{fig:pos2DforSpikes} marks with dots positions were spikes ocurred.

\begin{figure}
    \begin{center}
        \href{https://www.gatsby.ucl.ac.uk/~rapela/hippocampalDecoding/figures/pos2DForSpikesFrom50To80.html}{\includegraphics[width=5in]{../../../replay_trajectory_classification_test/figures/pos2DForSpikesFrom50To80.png}}
		\caption{Spikes fired by 104 neurons. A colored dot indicates the position at which the colored-matched neuron fired a spike. Click on the figure to access its interactive version.}
        \label{fig:pos2DforSpikes}
    \end{center}
\end{figure}

\section{Results}

We used the decoding algorithm by
\citep{denovellisEtAl21}\footnote{\url{https://github.com/Eden-Kramer-Lab/replay_trajectory_classification}}
on sorted spikes.

This algorithm first estimates the place fields of the neurons
(Figure~\ref{fig:placeFields}) and then use them to decode subject positions
from spiking neural activity (Figure~\ref{fig:decodings}).

\begin{figure}
    \begin{center}
        \href{https://www.gatsby.ucl.ac.uk/~rapela/hippocampalDecoding/figures/Jaq_03_16_sorted_spike_times_model_00000000_placeFields.html}{\includegraphics[width=5in]{../../../replay_trajectory_classification_test/figures/Jaq_03_16_sorted_spike_times_model_00000000_placeFields.png}}
		\caption{Place fields in linearized positions from 104 hippocampal neurons of the subject rat. Click on the figure to access its interactive version.}
        \label{fig:placeFields}
    \end{center}
\end{figure}

\begin{figure}
    \begin{center}
        \href{https://www.gatsby.ucl.ac.uk/~rapela/hippocampalDecoding/figures/Jaq_03_16_sorted_spike_times_00000000_decoding.html}{\includegraphics[width=5in]{../../../replay_trajectory_classification_test/figures/Jaq_03_16_sorted_spike_times_model_00000000_decoding.png}}
		\caption{Decodings. Red-yellow traces indicate time and linearized positions bins with larger probability. The cyan curve shows the subject positons. Click on the figure to access its interactive version.}
        \label{fig:decodings}
    \end{center}
\end{figure}

\bibliographystyle{apalike}
\bibliography{other}
\end{document}
